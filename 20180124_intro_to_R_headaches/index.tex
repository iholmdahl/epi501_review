\documentclass[]{article}
\usepackage{lmodern}
\usepackage{amssymb,amsmath}
\usepackage{ifxetex,ifluatex}
\usepackage{fixltx2e} % provides \textsubscript
\ifnum 0\ifxetex 1\fi\ifluatex 1\fi=0 % if pdftex
  \usepackage[T1]{fontenc}
  \usepackage[utf8]{inputenc}
\else % if luatex or xelatex
  \ifxetex
    \usepackage{mathspec}
  \else
    \usepackage{fontspec}
  \fi
  \defaultfontfeatures{Ligatures=TeX,Scale=MatchLowercase}
\fi
% use upquote if available, for straight quotes in verbatim environments
\IfFileExists{upquote.sty}{\usepackage{upquote}}{}
% use microtype if available
\IfFileExists{microtype.sty}{%
\usepackage{microtype}
\UseMicrotypeSet[protrusion]{basicmath} % disable protrusion for tt fonts
}{}
\usepackage[margin=1in]{geometry}
\usepackage{hyperref}
\hypersetup{unicode=true,
            pdftitle={Intro to R, plus some modeling},
            pdfauthor={Inga Holmdahl},
            pdfborder={0 0 0},
            breaklinks=true}
\urlstyle{same}  % don't use monospace font for urls
\usepackage{color}
\usepackage{fancyvrb}
\newcommand{\VerbBar}{|}
\newcommand{\VERB}{\Verb[commandchars=\\\{\}]}
\DefineVerbatimEnvironment{Highlighting}{Verbatim}{commandchars=\\\{\}}
% Add ',fontsize=\small' for more characters per line
\usepackage{framed}
\definecolor{shadecolor}{RGB}{248,248,248}
\newenvironment{Shaded}{\begin{snugshade}}{\end{snugshade}}
\newcommand{\KeywordTok}[1]{\textcolor[rgb]{0.13,0.29,0.53}{\textbf{#1}}}
\newcommand{\DataTypeTok}[1]{\textcolor[rgb]{0.13,0.29,0.53}{#1}}
\newcommand{\DecValTok}[1]{\textcolor[rgb]{0.00,0.00,0.81}{#1}}
\newcommand{\BaseNTok}[1]{\textcolor[rgb]{0.00,0.00,0.81}{#1}}
\newcommand{\FloatTok}[1]{\textcolor[rgb]{0.00,0.00,0.81}{#1}}
\newcommand{\ConstantTok}[1]{\textcolor[rgb]{0.00,0.00,0.00}{#1}}
\newcommand{\CharTok}[1]{\textcolor[rgb]{0.31,0.60,0.02}{#1}}
\newcommand{\SpecialCharTok}[1]{\textcolor[rgb]{0.00,0.00,0.00}{#1}}
\newcommand{\StringTok}[1]{\textcolor[rgb]{0.31,0.60,0.02}{#1}}
\newcommand{\VerbatimStringTok}[1]{\textcolor[rgb]{0.31,0.60,0.02}{#1}}
\newcommand{\SpecialStringTok}[1]{\textcolor[rgb]{0.31,0.60,0.02}{#1}}
\newcommand{\ImportTok}[1]{#1}
\newcommand{\CommentTok}[1]{\textcolor[rgb]{0.56,0.35,0.01}{\textit{#1}}}
\newcommand{\DocumentationTok}[1]{\textcolor[rgb]{0.56,0.35,0.01}{\textbf{\textit{#1}}}}
\newcommand{\AnnotationTok}[1]{\textcolor[rgb]{0.56,0.35,0.01}{\textbf{\textit{#1}}}}
\newcommand{\CommentVarTok}[1]{\textcolor[rgb]{0.56,0.35,0.01}{\textbf{\textit{#1}}}}
\newcommand{\OtherTok}[1]{\textcolor[rgb]{0.56,0.35,0.01}{#1}}
\newcommand{\FunctionTok}[1]{\textcolor[rgb]{0.00,0.00,0.00}{#1}}
\newcommand{\VariableTok}[1]{\textcolor[rgb]{0.00,0.00,0.00}{#1}}
\newcommand{\ControlFlowTok}[1]{\textcolor[rgb]{0.13,0.29,0.53}{\textbf{#1}}}
\newcommand{\OperatorTok}[1]{\textcolor[rgb]{0.81,0.36,0.00}{\textbf{#1}}}
\newcommand{\BuiltInTok}[1]{#1}
\newcommand{\ExtensionTok}[1]{#1}
\newcommand{\PreprocessorTok}[1]{\textcolor[rgb]{0.56,0.35,0.01}{\textit{#1}}}
\newcommand{\AttributeTok}[1]{\textcolor[rgb]{0.77,0.63,0.00}{#1}}
\newcommand{\RegionMarkerTok}[1]{#1}
\newcommand{\InformationTok}[1]{\textcolor[rgb]{0.56,0.35,0.01}{\textbf{\textit{#1}}}}
\newcommand{\WarningTok}[1]{\textcolor[rgb]{0.56,0.35,0.01}{\textbf{\textit{#1}}}}
\newcommand{\AlertTok}[1]{\textcolor[rgb]{0.94,0.16,0.16}{#1}}
\newcommand{\ErrorTok}[1]{\textcolor[rgb]{0.64,0.00,0.00}{\textbf{#1}}}
\newcommand{\NormalTok}[1]{#1}
\usepackage{longtable,booktabs}
\usepackage{graphicx,grffile}
\makeatletter
\def\maxwidth{\ifdim\Gin@nat@width>\linewidth\linewidth\else\Gin@nat@width\fi}
\def\maxheight{\ifdim\Gin@nat@height>\textheight\textheight\else\Gin@nat@height\fi}
\makeatother
% Scale images if necessary, so that they will not overflow the page
% margins by default, and it is still possible to overwrite the defaults
% using explicit options in \includegraphics[width, height, ...]{}
\setkeys{Gin}{width=\maxwidth,height=\maxheight,keepaspectratio}
\IfFileExists{parskip.sty}{%
\usepackage{parskip}
}{% else
\setlength{\parindent}{0pt}
\setlength{\parskip}{6pt plus 2pt minus 1pt}
}
\setlength{\emergencystretch}{3em}  % prevent overfull lines
\providecommand{\tightlist}{%
  \setlength{\itemsep}{0pt}\setlength{\parskip}{0pt}}
\setcounter{secnumdepth}{0}
% Redefines (sub)paragraphs to behave more like sections
\ifx\paragraph\undefined\else
\let\oldparagraph\paragraph
\renewcommand{\paragraph}[1]{\oldparagraph{#1}\mbox{}}
\fi
\ifx\subparagraph\undefined\else
\let\oldsubparagraph\subparagraph
\renewcommand{\subparagraph}[1]{\oldsubparagraph{#1}\mbox{}}
\fi

%%% Use protect on footnotes to avoid problems with footnotes in titles
\let\rmarkdownfootnote\footnote%
\def\footnote{\protect\rmarkdownfootnote}

%%% Change title format to be more compact
\usepackage{titling}

% Create subtitle command for use in maketitle
\newcommand{\subtitle}[1]{
  \posttitle{
    \begin{center}\large#1\end{center}
    }
}

\setlength{\droptitle}{-2em}

  \title{Intro to R, plus some modeling}
    \pretitle{\vspace{\droptitle}\centering\huge}
  \posttitle{\par}
    \author{Inga Holmdahl}
    \preauthor{\centering\large\emph}
  \postauthor{\par}
      \predate{\centering\large\emph}
  \postdate{\par}
    \date{2/1/2019}


\begin{document}
\maketitle

\section{Goals for today}\label{goals-for-today}

\begin{enumerate}
\def\labelenumi{\arabic{enumi}.}
\tightlist
\item
  Make sure everybody has \texttt{R}, RStudio, and \texttt{deSolve}
  installed and can run code
\end{enumerate}

--

\begin{enumerate}
\def\labelenumi{\arabic{enumi}.}
\setcounter{enumi}{1}
\tightlist
\item
  Go through discrete time model of headaches
\end{enumerate}

--

\begin{enumerate}
\def\labelenumi{\arabic{enumi}.}
\setcounter{enumi}{2}
\tightlist
\item
  Go through non-infectious ODE model of headaches
\end{enumerate}

--

\begin{enumerate}
\def\labelenumi{\arabic{enumi}.}
\setcounter{enumi}{3}
\tightlist
\item
  Go over questions from headaches worksheet
\end{enumerate}

\begin{center}\rule{0.5\linewidth}{\linethickness}\end{center}

class: center, middle, inverse \#\# Did anybody here have trouble
installing R?

\begin{center}\rule{0.5\linewidth}{\linethickness}\end{center}

class: center, middle, inverse \#\# Can you run this command with no
errors?

\begin{Shaded}
\begin{Highlighting}[]
\KeywordTok{library}\NormalTok{(deSolve)}
\end{Highlighting}
\end{Shaded}

\begin{longtable}[]{@{}l@{}}
\toprule
\begin{minipage}[t]{0.03\columnwidth}\raggedright\strut
If not, make sure you ran \texttt{install.packages("deSolve")}
first.\strut
\end{minipage}\tabularnewline
\bottomrule
\end{longtable}

class: center, middle, inverse \#\# Do you have the headache.zip file
from Canvas?

\section{Discrete model of headaches}\label{discrete-model-of-headaches}

\begin{itemize}
\tightlist
\item
  Here is \texttt{headache\_discrete.R} with comments removed:
\end{itemize}

\begin{Shaded}
\begin{Highlighting}[]
\NormalTok{N_t <-}\StringTok{ }\DecValTok{500}      
\NormalTok{P_t <-}\StringTok{ }\DecValTok{0}        
\NormalTok{incidence <-}\StringTok{ }\FloatTok{0.05} 
\NormalTok{recovery <-}\StringTok{ }\FloatTok{0.9}    

\NormalTok{dat <-}\StringTok{ }\OtherTok{NULL}
\NormalTok{timesteps <-}\StringTok{ }\DecValTok{1}\OperatorTok{:}\DecValTok{100}

\ControlFlowTok{for}\NormalTok{ (t }\ControlFlowTok{in}\NormalTok{ timesteps)\{}
\NormalTok{    N_t1 <-}\StringTok{ }\NormalTok{N_t }\OperatorTok{-}\StringTok{ }\KeywordTok{round}\NormalTok{(incidence}\OperatorTok{*}\NormalTok{N_t) }\OperatorTok{+}\StringTok{ }\KeywordTok{round}\NormalTok{(recovery}\OperatorTok{*}\NormalTok{P_t) }
\NormalTok{    P_t1 <-}\StringTok{ }\NormalTok{P_t }\OperatorTok{+}\StringTok{ }\KeywordTok{round}\NormalTok{(incidence}\OperatorTok{*}\NormalTok{N_t) }\OperatorTok{-}\StringTok{ }\KeywordTok{round}\NormalTok{(recovery}\OperatorTok{*}\NormalTok{P_t)}
\NormalTok{    dat <-}\StringTok{ }\KeywordTok{rbind}\NormalTok{(dat, }\KeywordTok{c}\NormalTok{(N_t1, P_t1)) }
    
\NormalTok{    N_t <-}\StringTok{ }\NormalTok{N_t1 }
\NormalTok{    P_t <-}\StringTok{ }\NormalTok{P_t1 }
\NormalTok{\}}

\KeywordTok{matplot}\NormalTok{(dat, }\DataTypeTok{xlab=}\StringTok{"time"}\NormalTok{, }\DataTypeTok{ylab=}\StringTok{"number of people"}\NormalTok{, }\DataTypeTok{type=}\StringTok{"l"}\NormalTok{, }\DataTypeTok{lty =} \DecValTok{1}\NormalTok{)}
\KeywordTok{legend}\NormalTok{(}\StringTok{"topright"}\NormalTok{, }\DataTypeTok{col =} \DecValTok{1}\OperatorTok{:}\DecValTok{2}\NormalTok{, }\DataTypeTok{legend =} \KeywordTok{c}\NormalTok{(}\StringTok{"no headache"}\NormalTok{,}\StringTok{"headache"}\NormalTok{), }\DataTypeTok{lwd=}\DecValTok{1}\NormalTok{)}
\end{Highlighting}
\end{Shaded}

\begin{itemize}
\tightlist
\item
  What's happening in this code? Line-by-line.
\end{itemize}

\section{Discrete model of
headaches}\label{discrete-model-of-headaches-1}

\begin{itemize}
\tightlist
\item
  What is the equilibrium prevalence of headaches in the population in
  this case?

  \begin{itemize}
  \tightlist
  \item
    prevalence = 27/500 = 0.054
  \end{itemize}
\end{itemize}

--

\begin{itemize}
\item
  Verify that \textbf{prevalence at equilibrium} is equal to the
  \textbf{incidence x duration}

  \begin{itemize}
  \item
    incidence = 0.05
  \item
    duration = 1/0.9
  \item
    1/0.9*0.05 = 0.055 --
  \item
    \emph{Note: this approximation only works at low prevalence}
  \end{itemize}
\end{itemize}

\section{Differential Equation model of
headaches}\label{differential-equation-model-of-headaches}

\begin{itemize}
\tightlist
\item
  Write them out.
\end{itemize}

\begin{Shaded}
\begin{Highlighting}[]
\KeywordTok{library}\NormalTok{(deSolve)}

\NormalTok{times <-}\StringTok{ }\KeywordTok{seq}\NormalTok{(}\DecValTok{0}\NormalTok{, }\DecValTok{5000}\NormalTok{, }\DataTypeTok{by =} \DecValTok{1}\NormalTok{)}
\NormalTok{yinit <-}\StringTok{ }\KeywordTok{c}\NormalTok{(}\DataTypeTok{no_headache =} \FloatTok{0.95}\NormalTok{, }\DataTypeTok{headache =} \FloatTok{0.05}\NormalTok{)}
\NormalTok{parameters <-}\StringTok{ }\KeywordTok{c}\NormalTok{(}\DataTypeTok{incidence =} \FloatTok{0.02}\NormalTok{, }\DataTypeTok{recovery =} \FloatTok{0.05}\NormalTok{)}

\NormalTok{headache_model <-}\StringTok{ }\ControlFlowTok{function}\NormalTok{(times, yinit, parameters) \{}
    \KeywordTok{with}\NormalTok{(}\KeywordTok{as.list}\NormalTok{(}\KeywordTok{c}\NormalTok{(yinit, parameters)), \{}
      
\NormalTok{     \{\{dno_headache <-}\StringTok{ }\NormalTok{recovery}\OperatorTok{*}\NormalTok{headache }\OperatorTok{-}\StringTok{ }\NormalTok{incidence}\OperatorTok{*}\NormalTok{no_headache\}\}}
\NormalTok{     \{\{dheadache <-}\StringTok{ }\NormalTok{incidence}\OperatorTok{*}\NormalTok{no_headache }\OperatorTok{-}\StringTok{ }\NormalTok{recovery}\OperatorTok{*}\NormalTok{headache\}\}}
      
\NormalTok{      comparts <-}\StringTok{ }\KeywordTok{list}\NormalTok{(}\KeywordTok{c}\NormalTok{(dno_headache, dheadache))}
      
      \KeywordTok{return}\NormalTok{(comparts)}
\NormalTok{    \})}
\NormalTok{\}}

\NormalTok{result <-}\StringTok{ }\KeywordTok{as.data.frame}\NormalTok{(}\KeywordTok{ode}\NormalTok{(}\DataTypeTok{y =}\NormalTok{ yinit, }\DataTypeTok{times =}\NormalTok{ times, }
                            \DataTypeTok{func =}\NormalTok{ headache_model, }\DataTypeTok{parms =}\NormalTok{ parameters))}

\KeywordTok{matplot}\NormalTok{(}\DataTypeTok{x =}\NormalTok{ result[, }\StringTok{"time"}\NormalTok{], }
        \DataTypeTok{y =}\NormalTok{ result[, }\KeywordTok{c}\NormalTok{(}\StringTok{"no_headache"}\NormalTok{, }\StringTok{"headache"}\NormalTok{)], }\DataTypeTok{type =} \StringTok{"l"}\NormalTok{)}
\end{Highlighting}
\end{Shaded}

\section{Differential Equation model of
headaches}\label{differential-equation-model-of-headaches-1}

\begin{itemize}
\tightlist
\item
  How would the model change if we add births and deaths?
\end{itemize}

\begin{Shaded}
\begin{Highlighting}[]
\KeywordTok{library}\NormalTok{(deSolve)}

\NormalTok{times <-}\StringTok{ }\KeywordTok{seq}\NormalTok{(}\DecValTok{0}\NormalTok{, }\DecValTok{5000}\NormalTok{, }\DataTypeTok{by =} \DecValTok{1}\NormalTok{)}
\NormalTok{yinit <-}\StringTok{ }\KeywordTok{c}\NormalTok{(}\DataTypeTok{no_headache =} \FloatTok{0.95}\NormalTok{, }\DataTypeTok{headache =} \FloatTok{0.05}\NormalTok{)}
\NormalTok{parameters <-}\StringTok{ }\KeywordTok{c}\NormalTok{(}\DataTypeTok{incidence =} \FloatTok{0.02}\NormalTok{, }\DataTypeTok{recovery =} \FloatTok{0.05}\NormalTok{)}

\NormalTok{headache_model <-}\StringTok{ }\ControlFlowTok{function}\NormalTok{(times, yinit, parameters) \{}
    \KeywordTok{with}\NormalTok{(}\KeywordTok{as.list}\NormalTok{(}\KeywordTok{c}\NormalTok{(yinit, parameters)), \{}
\NormalTok{      dno_headache <-}\StringTok{ }\NormalTok{recovery}\OperatorTok{*}\NormalTok{headache }\OperatorTok{-}\StringTok{ }\NormalTok{incidence}\OperatorTok{*}\NormalTok{no_headache}
\NormalTok{      dheadache <-}\StringTok{ }\NormalTok{incidence}\OperatorTok{*}\NormalTok{no_headache }\OperatorTok{-}\StringTok{ }\NormalTok{recovery}\OperatorTok{*}\NormalTok{headache}
\NormalTok{      comparts <-}\StringTok{ }\KeywordTok{list}\NormalTok{(}\KeywordTok{c}\NormalTok{(dno_headache, dheadache))}
      \KeywordTok{return}\NormalTok{(comparts)}
\NormalTok{    \})}
\NormalTok{\}}

\NormalTok{result <-}\StringTok{ }\KeywordTok{as.data.frame}\NormalTok{(}\KeywordTok{ode}\NormalTok{(}\DataTypeTok{y =}\NormalTok{ yinit, }\DataTypeTok{times =}\NormalTok{ times, }
                            \DataTypeTok{func =}\NormalTok{ headache_model, }\DataTypeTok{parms =}\NormalTok{ parameters))}

\KeywordTok{matplot}\NormalTok{(}\DataTypeTok{x =}\NormalTok{ result[, }\StringTok{"time"}\NormalTok{], }
        \DataTypeTok{y =}\NormalTok{ result[, }\KeywordTok{c}\NormalTok{(}\StringTok{"no_headache"}\NormalTok{, }\StringTok{"headache"}\NormalTok{)], }\DataTypeTok{type =} \StringTok{"l"}\NormalTok{)}
\end{Highlighting}
\end{Shaded}

\begin{center}\rule{0.5\linewidth}{\linethickness}\end{center}

\section{Differential Equation model of
headaches}\label{differential-equation-model-of-headaches-2}

\begin{itemize}
\tightlist
\item
  How would the model change if we add births and deaths?
\end{itemize}

\begin{Shaded}
\begin{Highlighting}[]
\KeywordTok{library}\NormalTok{(deSolve)}

\NormalTok{times <-}\StringTok{ }\KeywordTok{seq}\NormalTok{(}\DecValTok{0}\NormalTok{, }\DecValTok{5000}\NormalTok{, }\DataTypeTok{by =} \DecValTok{1}\NormalTok{)}
\NormalTok{yinit <-}\StringTok{ }\KeywordTok{c}\NormalTok{(}\DataTypeTok{no_headache =} \FloatTok{0.95}\NormalTok{, }\DataTypeTok{headache =} \FloatTok{0.05}\NormalTok{)}
\NormalTok{\{\{parameters <-}\StringTok{ }\KeywordTok{c}\NormalTok{(}\DataTypeTok{incidence =} \FloatTok{0.02}\NormalTok{, }\DataTypeTok{recovery =} \FloatTok{0.05}\NormalTok{, }\DataTypeTok{b =} \FloatTok{0.01}\NormalTok{, }\DataTypeTok{d =} \FloatTok{0.01}\NormalTok{)\}\}}

\NormalTok{headache_model <-}\StringTok{ }\ControlFlowTok{function}\NormalTok{(times, yinit, parameters) \{}
    \KeywordTok{with}\NormalTok{(}\KeywordTok{as.list}\NormalTok{(}\KeywordTok{c}\NormalTok{(yinit, parameters)), \{}
\NormalTok{      \{\{dno_headache <-}\StringTok{ }\NormalTok{recovery}\OperatorTok{*}\NormalTok{headache }\OperatorTok{-}\StringTok{ }\NormalTok{incidence}\OperatorTok{*}\NormalTok{no_headache }\OperatorTok{+}\StringTok{ }\NormalTok{b }\OperatorTok{-}\StringTok{ }
\StringTok{       }\NormalTok{d}\OperatorTok{*}\NormalTok{dno_headache\}\}}
\NormalTok{     \{\{dheadache <-}\StringTok{ }\NormalTok{incidence}\OperatorTok{*}\NormalTok{no_headache }\OperatorTok{-}\StringTok{ }\NormalTok{recovery}\OperatorTok{*}\NormalTok{headache }\OperatorTok{-}\StringTok{ }\NormalTok{d}\OperatorTok{*}\NormalTok{dheadache\}\}}
\NormalTok{      comparts <-}\StringTok{ }\KeywordTok{list}\NormalTok{(}\KeywordTok{c}\NormalTok{(dno_headache, dheadache))}
      \KeywordTok{return}\NormalTok{(comparts)}
\NormalTok{    \})}
\NormalTok{\}}

\NormalTok{result <-}\StringTok{ }\KeywordTok{as.data.frame}\NormalTok{(}\KeywordTok{ode}\NormalTok{(}\DataTypeTok{y =}\NormalTok{ yinit, }\DataTypeTok{times =}\NormalTok{ times, }
                            \DataTypeTok{func =}\NormalTok{ headache_model, }\DataTypeTok{parms =}\NormalTok{ parameters))}

\KeywordTok{matplot}\NormalTok{(}\DataTypeTok{x =}\NormalTok{ result[, }\StringTok{"time"}\NormalTok{], }
        \DataTypeTok{y =}\NormalTok{ result[, }\KeywordTok{c}\NormalTok{(}\StringTok{"no_headache"}\NormalTok{, }\StringTok{"headache"}\NormalTok{)], }\DataTypeTok{type =} \StringTok{"l"}\NormalTok{)}
\end{Highlighting}
\end{Shaded}

\begin{itemize}
\tightlist
\item
  \emph{Note: we just use \texttt{b} here because the total population
  size is \texttt{1}}
\end{itemize}

\begin{center}\rule{0.5\linewidth}{\linethickness}\end{center}

\section{Differential Equation model of
headaches}\label{differential-equation-model-of-headaches-3}

\begin{itemize}
\tightlist
\item
  Verify that \textbf{prevalence at equilibrium} is equal to the
  \textbf{incidence x duration}

  \begin{itemize}
  \item
    incidence = 0.02
  \item
    duration = 1/0.05
  \item
    incidence x duration = 0.4
  \item
    prevalence (\texttt{result\$headache{[}100{]}}) = 0.285
  \end{itemize}
\item
  This doesn't really work because of the high prevalence under these
  parameters
\end{itemize}

\begin{center}\rule{0.5\linewidth}{\linethickness}\end{center}

class: center, middle, inverse \# Any other questions?


\end{document}
